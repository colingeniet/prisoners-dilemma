\documentclass[10pt]{article}

\usepackage[utf8]{inputenc}
\usepackage[T1]{fontenc}
\usepackage[french]{babel}

\usepackage[table]{xcolor}

\title{Projet Architecture --- Système}
\author{Yoan Geran \and Colin Geniet}
\begin{document}
\maketitle
\tableofcontents

\section*{Introduction}

\section{Dilemme du prisonnier itéré}
\subsection{Usage}
\begin{verbatim}
$ make iterated_prisoner
$ ./iterated_prisoner <iterations>
\end{verbatim}

Ce programme simule toutes les combinaisons des 11 stratégies standard pour \verb|<iterations>| itérations,
et crée \verb|iterated_dilemma.tex| contenant un tableau des scores cumulés.
Voir \ref{strat_table} pour un exemple de résultat.

Le programme tente de compiler le fichier \LaTeX{} avec \verb|pdflatex|, et de l'afficher avec \verb|evince|
(modifiable dans \verb|latex_output.h|).

\begin{table}
\caption[Comparaison des stratégies]{Comparaisons des stratégies sur le dilemme du prisonnier itéré.}
\label{strat_table}
\begin{center}
\rowcolors{2}{white}{lightgray}
\begin{tabular}{|c|cccccc|}
\hline
& gent & méch & d-d & méf & d-d dur & ranc \\ \hline
gentille & 3000& 0& 3000& 2997& 3000& 3000\\
méchante & 5000& 1000& 1004& 1000& 1004& 1004\\
donnant-donnant & 3000& 999& 3000& 2500& 3000& 3000\\
méfiante & 3002& 1000& 2500& 1000& 1003& 1003\\
donnant-donnant-dur & 3000& 999& 3000& 1003& 3000& 3000\\
rancunière & 3000& 999& 3000& 1003& 3000& 3000\\
périodique gentille & 3666& 333& 2667& 2664& 1671& 343\\
périodique méchante & 4334& 667& 2003& 1999& 671& 671\\
majorité mou & 3000& 999& 3000& 2500& 3000& 3000\\
majorité dur & 3002& 1000& 2500& 1000& 1003& 1003\\
sondeur & 4996& 999& 1006& 1003& 1006& 1006\\
\hline
\end{tabular}
\\
\rowcolors{2}{white}{lightgray}
\begin{tabular}{|c|cccccc|}
\hline
& p gent & p méch & maj mou & maj dur & sond & total \\ \hline
gentille & 2001& 999& 3000& 2997& 3& 23997\\
méchante & 3668& 2332& 1004& 1000& 1004& 19020\\
donnant-donnant & 2667& 1998& 3000& 2500& 1001& 26665\\
méfiante & 2669& 1999& 2500& 1000& 1003& 18679\\
donnant-donnant-dur & 3331& 2331& 3000& 1003& 1001& 24668\\
rancunière & 3663& 2331& 3000& 1003& 1001& 25000\\
périodique gentille & 2334& 1665& 3666& 3663& 2660& 25332\\
périodique méchante & 3335& 1666& 671& 667& 1999& 18683\\
majorité mou & 2001& 2331& 3000& 2500& 1000& 26331\\
majorité dur & 2003& 2332& 2500& 1000& 1003& 18346\\
sondeur & 2670& 1998& 2503& 1003& 1002& 19192\\
\hline
\end{tabular}
\end{center}
Score cumulé du joueur 1 sur 1000 itérations.

lignes : joueur 1 --- colonnes : joueur 2
\end{table}

\subsection{Conception}
Cette première partie pose assez peu de problèmes de conception, 
le principal étant celui de la représentation des stratégies.
Nous avons décidé de les représenter par une fonction de décision, qui choisit une action
(coopérer ou trahir) selon ce qui s'est passé lors des parties précédentes.

Pour pouvoir manipuler les stratégies comme paramètre dans des fonctions d'ordre supérieur, 
nous avons choisit la signature commune suivante pour toutes les stratégies :
\begin{verbatim}
action (*strategy)(int n_played, action *hist, int n_coop)
\end{verbatim}

Les paramètres sont :
\begin{itemize}
\item \verb|n_played| : le nombre de parties jouées.
\item \verb|hist| : l'historique des actions de l'adversaire.
\item \verb|n_coop| : le nombre de parties durant lesquelles l'adversaire a coopéré.
\end{itemize}
Le choix de ces paramètres permet d'implémenter efficacement (en temps constant)
toutes les stratégies proposées. Bien sûr, \verb|n_coop| est redondant puisque
l'on peut le calculer à partir de \verb|hist|, mais cela impliquerait un surcout
considérable sur un nombre important d'itérations.
On pourrait également réduire \verb|hist|, car les stratégies implémentées n'en
utilisent qu'une partie réduite, mais cela limiterait alors sévèrement la
création de nouvelles stratégies.

\section{Simulation écologique}

\section{Simulation distribuée}

\end{document}
