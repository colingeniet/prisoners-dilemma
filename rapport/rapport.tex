\documentclass[10pt]{article}

\usepackage[utf8]{inputenc}
\usepackage[T1]{fontenc}
\usepackage[french]{babel}

\usepackage[table]{xcolor}

\title{Projet Architecture --- Système}
\author{Yoan Geran \and Colin Geniet}
\begin{document}
\maketitle
\tableofcontents

\section*{Introduction}

\section{Dilemme du prisonnier itéré}
\subsection{Usage}
\begin{verbatim}
$ make iterated_prisoner
$ ./iterated_prisoner <iterations>
\end{verbatim}

Ce programme simule toutes les combinaisons des 11 stratégies standard pour \verb|<iterations>| itérations,
et crée \verb|iterated_dilemma.tex| contenant un tableau des scores cumulés.
Voir \ref{strat_table} pour un exemple de résultat.

Le programme tente de compiler le fichier \LaTeX{} avec \verb|pdflatex|, et de l'afficher avec \verb|evince|
(modifiable dans \verb|latex_output.h|).

\begin{table}
\caption[Comparaison des stratégies]{Comparaisons des stratégies sur le dilemme du prisonnier itéré}
\label{strat_table}
\begin{center}
\rowcolors{2}{white}{lightgray}
\begin{tabular}{|c|cccccc|}
\hline
& gent & méch & d-d & méf & d-d dur & ranc \\ \hline
gentille & 3000& 0& 3000& 2997& 3000& 3000\\
méchante & 5000& 1000& 1004& 1000& 1004& 1004\\
donnant-donnant & 3000& 999& 3000& 2500& 3000& 3000\\
méfiante & 3002& 1000& 2500& 1000& 1003& 1003\\
donnant-donnant-dur & 3000& 999& 3000& 1003& 3000& 3000\\
rancunière & 3000& 999& 3000& 1003& 3000& 3000\\
périodique gentille & 3666& 333& 2667& 2664& 1671& 343\\
périodique méchante & 4334& 667& 2003& 1999& 671& 671\\
majorité mou & 3000& 999& 3000& 2500& 3000& 3000\\
majorité dur & 3002& 1000& 2500& 1000& 1003& 1003\\
sondeur & 4996& 999& 1006& 1003& 1006& 1006\\
\hline
\end{tabular}
\\
\rowcolors{2}{white}{lightgray}
\begin{tabular}{|c|cccccc|}
\hline
& p gent & p méch & maj mou & maj dur & sond & total \\ \hline
gentille & 2001& 999& 3000& 2997& 3& 23997\\
méchante & 3668& 2332& 1004& 1000& 1004& 19020\\
donnant-donnant & 2667& 1998& 3000& 2500& 1001& 26665\\
méfiante & 2669& 1999& 2500& 1000& 1003& 18679\\
donnant-donnant-dur & 3331& 2331& 3000& 1003& 1001& 24668\\
rancunière & 3663& 2331& 3000& 1003& 1001& 25000\\
périodique gentille & 2334& 1665& 3666& 3663& 2660& 25332\\
périodique méchante & 3335& 1666& 671& 667& 1999& 18683\\
majorité mou & 2001& 2331& 3000& 2500& 1000& 26331\\
majorité dur & 2003& 2332& 2500& 1000& 1003& 18346\\
sondeur & 2670& 1998& 2503& 1003& 1002& 19192\\
\hline
\end{tabular}
\end{center}
Score cumulé du joueur 1 sur 1000 itérations.

lignes : joueur 1 --- colonnes : joueur 2
\end{table}

\section{Simulation écologique}

\section{Simulation distribuée}

\end{document}
